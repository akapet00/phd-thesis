\cleardoublepage
\pagestyle{empty}

\section*{Curriculum Vitae}
\vspace{15mm}
\noindent \textbf{Ante Kapetanović} was born in Split, Croatia, in 1995.
He graduated from the Grammar School for Natural Sciences and Mathematics in Split, Croatia and subsequently pursued his undergraduate degree in Electrical Engineering in 2014 at the Faculty of Electrical Engineering.
In 2017, he furthered his studies by enrolling in the graduate program for Electronics and Computer Engineering.
Ante successfully completed his M.S. degree in Electrical Engineering at the Faculty of Electrical Engineering, Mechanical Engineering, and Naval Architecture, University of Split, Split, Croatia, in 2019.
To date, he has been working toward the Ph.D. degree in computational bioelectromagnetics with the University of Split.

Ante did a three-month research visit at the Aalborg University, Aalborg, Denmark during his M.S.
Additionally, he participated in two short-term research visits at the IETR/CNRS in Rennes, France in 2021 and 2022.

His research interests revolve around human exposure to electromagnetic fields, computational bioelectromagnetics, numerical methods and techniques, physics-informed machine learning and scientific computing in general.
He has authored and co-authored six journal and more than fifteen conference papers in various fields of computational science.
His research was recognized at the 2022 IEEE MTT-S International Microwave Biomedical Conference in Sozhou, China, where he received the overall Best Student Paper Award.

Ante actively participates in professional societies and associations.
He has been a member of the Croatian Chapter of the IEEE Electromagnetic Compatibility Society since 2020 and a Student Member of the BioEM (previously European Bioelectromagnetics Association (EBEA) before a merger with The Bioelectromagnetics Society (BEMS)) since 2021.
Currently, he is an active member of IEEE Working Group on power density averaging methods within International Committee on Electromagnetic Safety (ICES), Technical Committee 95, Sub-Committee 6 on electromagnetic field dosimetry modeling.

\cleardoublepage
\pagestyle{empty}
\section*{Životopis}
\vspace{15mm}
\noindent
\textbf{Ante Kapetanović} rođen je u Splitu 1995. godine.
Po završetku srednjoškolskog obrazovanja kojeg je stekao u III. gimnaziji Split (Prirodoslovno-matematička gimnazija u Splitu, popularni MIOC), 2014. godine je upisao preddiplomski studij Elektrotehnike i informacijske tehnologije, a 2017. godine i diplomski studij Elektronike i računalnog inženjerstva.
Titulu magistra inženjera elektrotehnike je stekao 2019. godine na Fakultetu elektrotehnike, strojarstva i brodogradnje pri Sveučilištu u Splitu.
Od kraja 2019. godine, zaposlen je kao mlađi istraživač u području računalnog bioelektromagnetizma na Sveučilištu u Splitu.

Ante je boravio tri mjeseca na istraživačkom posjetu na Sveučilištu Aalborg, Aalborg, Danska, tijekom diplomskog studija.
Tijekom doktorata, dva puta posjećuje institut IETR/CNRS, Rennes, Francuska, u sklopu suradnje na znanstvenim projektima.

Njegovi istraživački interesi su izloženost ljudi elektromagnetskim poljima, računalni bioelektromagnetizam, numeričke metode i tehnike, strojno učenje utemeljeno na fizici i znanstveno računanje općenito.
Do danas je autor i suator šest radova u časopisima i više od petnaest konferencijskih radova iz različitih područja računalnih znanosti.
Njegovo je istraživanje prepoznato na međunarodnoj konferenciji IEEE MTT-S u Sozhouu, Kina, gdje je 2022. godine dobio nagradu za najbolji studentski rad.

Ante je član hrvatskog ogranka IEEE društva za elektromagnetsku kompatibilnost od 2020. i student-član BioEM društva (prethodno Europske udruge za bioelektromagnetizam (EBEA) prije spajanja s Bioelektromagnetskim društvom (BEMS)) od 2021. godine.
Također je član i IEEE Radne skupine za metode određivanja gustoće snage, Tehničkog odbora 95 (Pododbor 6 za modeliranje dozimetrije elektromagnetskog polja), Međunarodnog odbora za elektromagnetsku sigurnost (ICES).
\newpage