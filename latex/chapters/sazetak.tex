\setlength{\parindent}{0in}
{\fontsize{14}{18}\bf {Napredna tehnika određivanja prostorno usrednjenih dozimetrijskih veličina na zakrivljenim površinama}}

\vskip 15mm
\addcontentsline{toc}{section}{Sa\v{z}etak}
\textbf{Sa\v{z}etak:\\}
   
\textnormal{Pojava pete generacije mobilne mrežne tehnologije i razvoj povezanih komunikacijskih sustava dovela je do povećane izloženosti ljudi elektromagnetskim poljima visokih frekvencija.
Kako bi se osiguralo korištenje bežičnih komunikacijskih uređaja u neposrednoj blizini ljudskog tijela bez negativnih posljedica na zdravlje, regulatorni odbori su osnovani na međunarodnoj razini sa svrhom postavljanja najviše dopuštene razine izloženosti.
Međutim, većina istraživačkih i znanstvenih radova, usmjerenih na procjenu apsorbirane snage unutar tkiva, a na kojima se granice izloženosti temelje, koristi ravne modele za predstavljanje izloženih dijelova ljudskog tijela.
Ovakva aproksimacija geometrije potencijalno dovodi do poddimenzioniranja razine izloženosti, ovisno o omjeru dubine prodiranja elektromagnetskih polja i polumjera zakrivljenosti izloženih dijelova tijela.
Ovaj doktorski rad doprinosi području računalne dozimetrije kroz razvoj napredne tehnike prostornog usrednjavanja dozimetrijskih veličina na zakrivljenim površinama tijela, s posebnim naglaskom na frekvencije od \num{6} do \num{300} GHz.
Temeljni cilj istraživanja je kvantifikacija učinka površinske zakrivljenosti, osobito u slučaju kada valna duljina upadnih polja veličinom odgovara približnom polumjeru zakrivljenosti.
Razvijena su dva kanonska modela---kugla (članak 1) i cilindar (članak 2)---te detaljan anatomski model ljudskog uha (članak 3).
U svrhu što vjernije procjene usrednjene gustoće snage na zakrivljenim modelima, predstavljena je i nova tehnika numeričke integracije, koja posredno ostvaruje otkrivanje ograničenog područja najviše izloženosti.
Nadalje, strojno učenje i povezane tehnike iskorištene su za unaprjeđenje učinkovitosti i smanjenje potrebe za računalnim resursima prilikom elektromagnetskih simulacija (članci 2 i 3).
Konačno, rad dublje zadire i u same tehnike numeričke integracije namjenjene aproksimaciji plošnih integrala po konformnim površinama na frekvencijama iznad \num{6} GHz (članak 4).
Istraživanje predstavljeno u okviru ovog doktorskog rada proširuje razumijevanje ljudske izloženosti elektromagnetskim poljima radijskih frekvencija i doprinosi razvoju elektromagnetskih modela izloženosti prilagođenih kontekstu nadolazećih bežičnih komunikacijskih tehnologija.}


\vskip 15mm
\bf{Ključne riječi:\\}
\textnormal{elektromagnetska sigurnost, procjena izloženosti, računalna dozimetrija, gustoća apsorbirane snage, gustoća upadne snage, zakrivljena površina, anatomski modeli, procjena normala, plošna integracija, strojno učenje}
