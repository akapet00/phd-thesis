\setlength{\parindent}{0in}
{\fontsize{14}{18}\bf {Advanced Technique for Assessment of Spatially Averaged Dosimetric Quantities on Nonplanar Surfaces}}

\vskip 15mm
\addcontentsline{toc}{section}{Abstract}
\textbf{Abstract:\\}	

\textnormal{With the rapid expansion of the fifth generation wireless communication technology and systems, human exposure to radio-frequency electromagnetic fields has become increasingly prevalent.
International regulatory institutions have been established to ensure the safe use of these devices by setting maximum allowable levels of exposure.
However, the existing body of literature in computational dosimetry predominantly rely on simplified models that employ flat surfaces to represent the human body.
This geometrical approximation may lead to inaccurate estimation of exposure, depending on the ratio of the penetration depth to the local curvature radius of a nonplanar body part that is being exposed.
Recognizing the aforementioned limitation, the primary objective of this thesis is to make contributions by advancing techniques for spatial averaging of dosimetric quantities on nonplanar surfaces, with a specific focus on the \SIrange{6}{300}{\GHz} frequency range.
The aim is to quantify the effect of surface curvature, especially in situations where the wavelength of the incident field matches the radius defining the local curvature.
Two canonical models, the sphere (Publication 1) and cylinder (Publication 2), and a detailed anatomical model of the human ear have been presented (Publication 3).
To assess spatially averaged power densities on these nonplanar models with high fidelity, a novel numerical surface integration technique is incorporated into the thesis.
This technique facilitates the identification of the region characterized by the worst-case exposure scenario.
Furthermore, the integration of machine learning techniques has shown promise in enhancing the accuracy, increasing the efficiency, and reducing the memory requirements during electromagnetic simulations, as demonstrated in Publications 2 and 3.
Finally, the thesis delves deeply into quadrature techniques specifically tailored for surface integrals on conformal surfaces at microwave and millimeter wave frequencies (Publication 4).
Overall, the research output presented within the thesis improves the understanding of human exposure to high-frequency electromagnetic fields and contribute to the development of more precise dosimetric models in the context of emerging wireless communication technologies.
}
	
\vskip 15mm
\bf{Keywords:\\}
\textnormal{electromagnetic safety, exposure assessment, computational dosimetry, absorbed power density, incident power density, nonplanar surface, anatomical models, normal estimation, surface integration, machine learning}
